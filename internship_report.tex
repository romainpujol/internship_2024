\documentclass{amsart}
\usepackage[T1]{fontenc}
\usepackage[utf8]{inputenc}
\usepackage{algorithm2e}
\usepackage{longtable}
\usepackage{fancyhdr}
\usepackage{amsmath}
\usepackage{amsfonts}
\usepackage{enumitem}
\usepackage{amssymb}
\usepackage[left=2cm,right=2cm,top=3cm,bottom=3cm]{geometry}
\usepackage{graphicx}
\usepackage{hyperref}
\usepackage{xcolor}
\usepackage{lastpage} 
\usepackage{setspace}
\usepackage{titlesec}
\usepackage{dsfont}
\usepackage{cleveref}
\usepackage{bm}
\usepackage[colorinlistoftodos,bordercolor=orange,backgroundcolor=orange!20,linecolor=orange,textsize=scriptsize]{todonotes}
\usepackage{subcaption}
\SetAlgoNlRelativeSize{-1}

\geometry{
    top=1in, % Marge supérieure
    bottom=1.5in, % Marge inférieure
    left=1in, % Marge gauche
    right=1in, % Marge droite
    footskip=1in % Espace entre le bas du texte principal et le pied de page
}


\pagestyle{fancy}
\fancyhf{} 

\fancypagestyle{plain}{
    \fancyhf{} 
}


\titleformat{\section}[hang]
  {\normalfont\Large\bfseries} % Format du texte
  {\thesection{} - } % Préfixe (ex. 1.1 - )
  {0pt} % Espace entre le préfixe et le titre
  {\Large\bfseries} % Format du titre

% Configurer les sous-sections
\titleformat{\subsection}[hang]
  {\normalfont\normalsize\bfseries} % Format du texte
  {\thesubsection{} - } % Préfixe (ex. 1.1.1 - )
  {0pt} % Espace entre le préfixe et le titre
  {\normalsize\bfseries} % Format du titre

% Configurer les sous-sous-sections
\titleformat{\subsubsection}[hang]
  {\normalfont\small\bfseries} % Format du texte
  {\thesubsubsection{} - } % Préfixe (ex. 1.1.1.1 - )
  {0pt} % Espace entre le préfixe et le titre
  {\small\bfseries} % Format du titre
  
\ifdefined\theorem\else \newtheorem{theorem}{Theorem}\fi
\ifdefined\proposition\else \newtheorem{proposition}[theorem]{Proposition}\fi
\ifdefined\definition\else \newtheorem{definition}[theorem]{Definition}\fi
\ifdefined\lemma\else \newtheorem{lemma}[theorem]{Lemma}\fi
\ifdefined\corollary\else \newtheorem{corollary}[theorem]{Corollary}\fi
\ifdefined\remark\else \newtheorem{remark}[theorem]{Remark}\fi
\ifdefined\assumption\else \newtheorem{assumption}{Assumption}\fi
\ifdefined\example\else \newtheorem{example}{Example}\fi


\newcommand{\argmin}{\mathop{\arg\min}}
\crefformat{enumi}{#2#1#3}
\newcommand{\nb}[3]{
		{\colorbox{#2}{\bfseries\sffamily\tiny\textcolor{white}{#1}}}
		{\textcolor{#2}{\text{$\blacktriangleright$}{\textcolor{#2}{#3}}\text{$\blacktriangleleft$}}}}
\newcommand{\rp}[1]{\nb{RP}{red}{#1}}
\newcommand{\af}[1]{\nb{AF}{blue}{#1}}
\newcommand{\bt}[1]{\nb{BT}{orange}{#1}}
\newcommand{\RR}{\mathbb{R}}

\renewcommand{\contentsname}{Table of contents}


\begin{document}


\thispagestyle{empty}

\begin{figure}[htbp]
    \centering
    \begin{minipage}[b]{0.22\textwidth}
        \centering
        \includegraphics[width=\textwidth]{logo/logo.jpg}
    \end{minipage}
    \hfill
    \begin{minipage}[b]{0.25\textwidth}
        \centering
        \includegraphics[width=\textwidth]{logo/universita di pisa.png}
    \end{minipage}
\end{figure}

\vspace{1.5cm}	

\begin{center}	
    {\huge \bf\ Research Internship  \\
    \vspace{1cm}
    \huge Scenario reduction techniques for two-stage programming. \\
    \vspace{2cm}

    \large Field of study : Applied Mathematics \\
    \large School year : 2023 - 2024 \\



    \vspace{3.5cm}
    \large\textcolor{red}{Confidentiality notice : }\\
    \color{black}
    \vspace{0.5cm}
    }
\end{center}

\noindent\large \textbf{Author} Romain Pujol  \\
\textbf{Class} 2025 \\

\noindent\textbf{ENSTA Paris supervisor} Sorin-Mihai Grad \\ 
\textbf{Università di Pisa supervisor} Antonio Frangioni \\

\noindent Internship carried out from 13/05/2024 to 02/08/2024. 
\\
Università di Pisa, Lungarno Antonio Pacinotti, 43.


\newpage
\thispagestyle{empty}
\mbox{}
\newpage



%%% PAGE DE CONFIDENTIALITE
%%% PAGE DE CONFIDENTIALITE
%%% PAGE DE CONFIDENTIALITE
%%% PAGE DE CONFIDENTIALITE
%%% PAGE DE CONFIDENTIALITE
%%% PAGE DE CONFIDENTIALITE
%%% PAGE DE CONFIDENTIALITE



\begin{titlepage}

    \vspace*{5cm}
        \Large\textbf{Acknowledgements}
    \vspace{1cm}
    
    \normalsize Thanks.
\end{titlepage}

\newpage

\begin{titlepage}

    \vspace*{1cm}
        \Large\textbf{Abstract} \\
    
    \normalsize Two-stage stochastic optimization problems become very large when the distributions involved have a large number of atoms. The goal is to solve simplified two-stage problems by reducing the size of the initial distribution in order to decrease CPU time. This scenario reduction is ruled by a certain metric, the Wasserstein distance. It is also necessary to ensure that the optimal values obtained from both problems are similar. In order to integrate scenario reduction techniques into the SMS++ ecosystem at the University of Pisa, a special attention is paid to algorithmic efficiency. \\ 
    
    \textbf{Key-words:} Stochastic programming, scenario reduction, Wasserstein distance, greedy algorithm, local-search, stability.
  \\

        \vspace*{1cm}
        \Large\textbf{Résumé} \\
    
    \normalsize Les problèmes d'optimisation stochastique à deux étapes sont de très grande taille lorsque les distributions mises en jeu ont un grand nombre d'atomes. L'objectif sera de résoudre des problèmes à deux étapes simplifiés en \emph{réduisant} la taille de la distribution initiale pour diminuer le temps de calcul. Cette réduction de scénarios est gouvernée par une certaine métrique, la distance de Wasserstein. Il faut aussi être capable de s'assurer que les valeurs optimales obtenues via les deux problèmes sont similaires. Dans le but d'intégrer les techniques de réduction de scénario dans l'écosystème SMS++ de l'Université de Pise, une grande attention est portée à l'efficacité algorithmique. \\
    
    \textbf{Mots-clés :} Optimisation stochastique, réduction de scénario, distance de Wasserstein, algorithme glouton, recherche locale, stabilité.
    
\end{titlepage}


% Table des matières
\tableofcontents

\newpage
\pagestyle{fancy} % Activer le style fancy
\fancyfoot[R]{\thepage/\pageref{LastPage}} 
\fancyfoot[C]{Romain Pujol / Università di Pisa \\ \textcolor{red}{Confidential / Non confidential report}}

\section{Stability in stochastic programming}\label{stability}
\subsection{Context and notions}
Many Stochastic Programming (SP) models can be written as:
\begin{equation}\label{stochastic}
\min\left\{\int_\Xi F_0\left(x,\xi\right)\text{d}P\left(\xi\right)\:,\: x\in X, \: \int_\Xi F_j\left(x,\xi\right)\text{d}P\left(\xi\right)\leq0, j=1,...,d\right\}.
\end{equation}

Here $X\subset\RR^m$ is closed, $\Xi\subset\RR^s$ is closed, $F_j$ are random lower semi-continuous functions for all $j\in\{0,...,d\}$ and $P$ is a Borel probability measure on $\Xi$. The set $X$ does not depend on $P$ and defines constraints on $x$. The goal of this part is to highlight a specific metric on distributions that would ensure that if two distributions $P$ and $Q$ are close, then the value of both SP models are close. We write:
\begin{align*}
    &\mathcal{X}\left(P\right)=\left\{x\in X, \int_\Xi F_j\left(x,\xi\right)\text{d}P\left(\xi\right)\leq0, j=1,...,d\right\},\\
    &v\left(P\right)=\inf\left\{\int_\Xi F_0\left(x,\xi\right)\text{d}P\left(\xi\right)\:,\: x\in \mathcal{X}\left(P\right)\right\}, \\
    &X^*\left(P\right)=\argmin\left\{\int_\Xi F_0\left(x,\xi\right)\text{d}P\left(\xi\right)\:,\: x\in X, \: \int_\Xi F_j\left(x,\xi\right)\text{d}P\left(\xi\right)\leq0, j=1,...,d\right\}.
\end{align*}
So $\mathcal{X}\left(P\right)$ denotes the feasible set, $v\left(P\right)$ the optimal value of \eqref{stochastic} and $X^*\left(P\right)$ its set of minimizers. For every integer $p\geq 1$, let us define $\mathcal{F}_p\left(\Xi\right)$ a class of measurable functions from $\Xi$ to $\RR$ named locally Lipschitz continuous functions:
\begin{align*}
    &\mathcal{F}_p\left(\Xi\right)=\left\{f:\Xi\to \RR: \lvert f\left(\xi\right)-f\left(\xi'\right)\rvert \leq c_p\left(\xi,\xi'\right)\lVert\xi-\xi'\rVert, \forall \xi,\xi'\in \Xi^2 \right\}, \\
   & c_p\left(\xi,\xi'\right)=\max \left\{1,\lVert\xi \rVert,\lVert\xi'\rVert\right\}^{p-1}.
\end{align*}
When $p=1$, we recall 1-Lipschitz functions. When $p\geq 2$, there exists another dependency on the space $\Xi$ allowing functions with higher and higher variations as $p$ increases. One can note that $F_1\left(\Xi\right)\subset F_2\left(\Xi\right)\subset F_3\left(\Xi\right) \hdots$ For $P,Q\in\mathcal{P}_p\left(\Xi\right)^2$ where $\mathcal{P}_p\left(\Xi\right)=\left\{Q\in\mathcal{P}\left(\Xi\right), \int_\Xi \lVert\xi\rVert^pdQ\left(\xi\right)\right\}$, the latest statement allows us to define $p$-th order Fortet-Mourier metric as:
$$
\zeta_p\left(P,Q\right)=\sup_{f\in\mathcal{F}_p\left(\Xi\right)}\lvert \int_\Xi f\left(\xi\right)\text{d}P\left(\xi\right)-\int_\Xi f\left(\xi\right)\text{d}Q\left(\xi\right)\rvert.
$$
See \href{https://www.imo.universite-paris-saclay.fr/~pierre-loic.meliot/master/exam-2017.pdf}{metric proof}.

\subsection{Result}
Let $Q\in\mathcal{P}\left(\Xi\right)$, we write the perturbed model:
\begin{equation*}
    \min\left\{\int_\Xi F_0\left(x,\xi\right)\text{d}Q\left(\xi\right)\:,\: x\in X, \: \int_\Xi F_j\left(x,\xi\right)\text{d}Q\left(\xi\right)\leq0, j=1,...,d\right\},
\end{equation*}
which is \eqref{stochastic} but with the distribution $Q$ instead of $P$. In this part we will try to compare the notions we have presented in the above part, for instance, we want to bound $\lvert v\left(P\right)-v\left(Q\right)\rvert$ in terms of distribution metrics. \cite[Corollary 14]{romisch_stability_2003} gives:
\begin{theorem}\label{stability_th}
    Let the number of constraints in \eqref{stochastic} be $d=0$ and assume that:
    \begin{enumerate}
        \item The  solution set $X^*\left(P\right)$ is nonempty and $\mathcal{U}$ is an open, bounded neighbourhood of $X^*\left(P\right)$.
        \item The set $X$ is convex and $F_0\left(\cdot,\xi\right)$ is convex on $\RR^m$ for each $\xi\in\Xi$.
        \item there exist constants $L>0, p\geq1$ such that $\frac{1}{L}F_0\left(x,\cdot \right)\in\mathcal{F}_p\left(\Xi\right)$ for each $x\in X\cap cl\mathcal{U}$. 
    \end{enumerate}
    Then there exists a constant $\delta>0$ such that:
    \begin{align*}
        &\lvert v\left(P\right)-v\left(Q\right)\rvert \leq L\zeta_p\left(P,Q\right) \\
        & \emptyset \ne X^*\left(Q\right)\subset X^*\left(P\right)+\Psi_P\left(L\zeta_p\left(P,Q\right)\right)\mathbb{B}
    \end{align*}
    Where $\Psi_p$ is defined in \cite[2.22-2.23]{romisch_stability_2003}. Whenever $Q\in\mathcal{P}\left(\Xi\right)$ with finite $p$-th order absolute moments and $\zeta_p\left(P,Q\right)<\delta$
\end{theorem}
These results has two main points. First, we understand that under a slight perturbation of $P$ (in terms of Fortet-Mourier metric), the optimal value is controlled. The parameter $p$ appears in the third assumption. Second, the set of optimal $x$ of the perturbed model will remain close to the set of optimal $x$ of the original problem. This is a result of stability. 

\subsection{Application to two-stage linear models}\label{two stage}
We consider a linear two-stage model with fixed recourse, this model is generically written as: 
\begin{equation}\label{lp stoch}
    \min_x\left\{c^Tx + \mathbb{E}_\xi\left(\inf_y q\left(\xi\right)^Ty \right)\: :\: Wy\left(\xi\right)=h\left(\xi\right)-T\left(\xi\right)x\:,\: y\left(\xi\right)\geq0\:,\: x\in X \right\}.
\end{equation}
Where $x$ is called the first-stage decision variable, $y$ the second. The general idea of this model is a first-stage decision has to be made before observing $\xi$. The distribution of $\xi$ is known so that $x$ is chosen to ensure the best average outcome. Then after observing $\xi$, the second-stage variable $y$ is determined. One can note that the only decision you make is $x$ as $y$ is determined by $x$ and $\xi$. The term "fixed recourse" highlights the nature of $W$, a fixed matrix that does not depend (in that case) on $\xi$. See \cite{wets_stochastic_1974} for a complete survey on linear two-stage programs with fixed recourse. In order to fit in the definition  \ref{stochastic}, we define:
$$
F_0\left(x,\xi\right)=\begin{cases} 
  c^Tx + \phi\left(q\left(\xi\right), h\left(\xi\right) -T\left(\xi\right)x\right) & h\left(\xi\right)-     T\left(\xi\right)x \in \text{pos}\left(W\right), q\left(\xi\right) \in D, \\
  +\infty & \text{otherwise},
\end{cases}
$$
where $\text{pos}(W)=\left\{Wy, y\in\RR_+^m\right\}$, $D=\left\{u\in\RR^m:\left\{z\in\RR^r:W^Tz \leq u\right\}\ne \emptyset\right\}$ and  \\$\phi\left(u,t\right)=\inf\left\{ u^Ty : Wy=t, y\geq0\right\}$. \ref{lp stoch} can be rewritten as:
\begin{equation}\label{rewrite}
    \min_x\left\{\int_\Xi F_0\left(x,\xi\right)\text{d}P\left(\xi\right): x\in X\right\}. 
\end{equation}
One can note that $d=0$ in this formulation. Let us introduce two assumptions: 
\begin{assumption}\label{h1} For each $\left(x,\xi\right)\in X\times \Xi$ it holds that $h\left(\xi\right)- T\left(\xi\right)x\in \text{pos}\left(W\right), q\left(\xi\right) \in D$.
\end{assumption}
\begin{assumption}
\label{h2} $P\in\mathcal{P}\left(\Xi\right)$ and has a finite $2$-th moment i.e. $\int_\Xi \lVert \xi\rVert^2\text{d}P\left(\xi\right) < \infty$.
\end{assumption}
\noindent Under these two assumptions, equivalence between \ref{lp stoch} and \ref{rewrite} is direct, even though it may look harsh, as both terms inside the integral converge then it is a matter of notation.
%\begin{proof}
    %Let $x\in X$. \Cref{h1} states that $F_0$ will always be written $c^Tx + \phi\left(q\left(\xi\right), h\left(\xi\right) -T\left(\xi\right)x\right)$ when we integrate it on $\Xi$ when $x\in X$.
    %\begin{align*}
    %    \int_\Xi F_0\left(x,\xi\right)\text{d}P\left(\xi\right)&=\int_\Xi c^Tx + \phi\left(q\left(\xi\right), h\left(\xi\right) -T\left(\xi\right)x\right)\text{d}P\left(\xi\right) \\
    %    &= c^Tx + \int_\Xi \inf_y\left\{q\left(\xi\right)^Ty\: :\: Wy=h\left(\xi\right) -T\left(\xi\right)x\right\}\text{d}P\left(\xi\right) \\
    %    &= \left\{c^Tx + \int_\Xi \inf_y\left\{q\left(\xi\right)^Ty\right\}\text{d}P\left(\xi\right)\:,\: Wy=h\left(\xi\right) -T\left(\xi\right)x\right\}
    %\end{align*}
    %There is no problem to split the integral in two as $c^Tx$ is finite, does not depend on $\xi$ and the whole integral is finite aswell. We  used obviously $\int_\Xi\text{d}P\left(\xi\right)=1$. The third equality is a matter of notation. 
%\end{proof}

\begin{proposition}\label{prop 2}
    Let \Cref{h1} be satisfied. Then $F_0$ is a random convex function. Furthermore there exist constants $L>0, \bar{L}>0$, and $K>0$ such that the following holds for all $\xi,\xi'\in\Xi^2$ and $x,x'\in X^2$ with $\max\left\{\lVert x\rVert, \lVert x' \rVert\right\}\leq r$:
    \begin{align}
        \lvert F_0\left(x,\xi\right)- F_0\left(x,\xi'\right)\rvert &\leq Lr\max\left\{1,\lVert \xi\rVert, \lVert \xi'\rVert\right\}\lVert \xi-\xi'\rVert, \label{lips}\\
        \lvert F_0\left(x,\xi\right)- F_0\left(x',\xi\right)\rvert &\leq \bar{L}\max\left\{1,\lVert\xi\rVert^2\right\}\lVert x-x'\rVert, \\
        \lvert F_0\left(x,\xi\right) \rvert&\leq Kr\max\left\{1,\lVert\xi\rVert^2\right\}.
    \end{align}
     
\end{proposition}
\begin{proof}
    See \cite[Proposition 22]{romisch_stability_2003}.
\end{proof}
Thereafter we will work with finite distributions, meaning that $\Xi$ will always be bounded. With $\Xi$ be bounded, we can modify a little what is said in \Cref{lips} from \Cref{prop 2} to our best convenience:
$$
\lvert F_0\left(x,\xi\right)- F_0\left(x,\xi'\right) \leq L'r \lVert \lVert \xi-\xi'\rVert, \text{ with } L'=L\max\left\{1,\max_{\xi\in\Xi}\lVert\xi\rVert\right\}.
$$
This means that there exists a constant $C>0$ such that $\frac{1}{C}F_0\left(x,\cdot\right)\in\mathcal{F}_1\left(\Xi\right)$ for each $x\in X$ hence if we fix $U$ an open, bounded neighbourhood of $X^*\left(P\right)$ the result hold for each $x\in X\cap \text{cl}U\subset X$. In other words, condition (3) of \Cref{stability_th} is verified with $p=1$.

\begin{corollary}
    Let both assumptions \Cref{h1} and \Cref{h2} be satisfied, $\Xi$ be bounded and let $X^*\left(P\right)$ be nonempty and $\mathcal{U}$ be an open, bounded neighbourhood of $X^*\left(P\right).$ Then there exist constants $L>0$ and $\delta >0$ such that:
    \begin{align*}
        \lvert v\left(P\right)-v\left(Q\right)\rvert \leq L\zeta_1\left(P,Q\right) \\
        \emptyset \ne X^*\left(Q\right)\subset X^*\left(P\right)+\Psi_P\left(L\zeta_1\left(P,Q\right)\right)\mathbb{B}
    \end{align*}
    whenever $Q\in\mathcal{P}\left(\Xi\right)$ and has a finite $1$-th moment and $\zeta_1\left(P,Q\right)<\delta$.
\end{corollary}
\begin{proof}
    This is an application of \Cref{stability_th} with $p=1$ where both assumptions are sufficient to ensure that conditions (2), (3) of \Cref{stability_th} are verified. Condition (1) is verified when the problem is well posed, meaning that there exists a solution. See \cite[Proposition 22]{romisch_stability_2003}.
\end{proof}
\begin{remark}
    In \cite{romisch_stability_2003}, they do not assume that $\Xi$ is bounded and then the results hold with $p=2$.
\end{remark}
Consider now a two-stage model with a distribution $P=\sum_{i\in I}\delta_{x_i}p_i$ with $n$ atoms, $n$ is large. We may want to find another distribution $Q$ with a reduced number of atoms close to $P$ in terms of $\zeta_1$ so that $\lvert v\left(P\right)-v\left(Q\right)\rvert$ is controlled. Finding such a distribution $Q$ would reduce run time when solving the associated SP on a computer. Let us recall the definition of the $p$-th order Fortet-Mourier metric:
$$
\zeta_1\left(P,Q\right)=\sup_{F\in\mathcal{F}_1\left(\Xi\right)}\lvert \int_\Xi F\left(\xi\right)\text{d}P\left(\xi\right)-\int_\Xi F\left(\xi\right)\text{d}Q\left(\xi\right)\rvert,
$$
where $\mathcal{F}_1\left(\Xi\right)=\left\{F:\Xi\to \RR: \lvert F\left(\xi\right)-F\left(\tilde{\xi}\right)\rvert \leq \lVert\xi-\xi'\rVert, \forall \xi,\xi'\in \Xi^2 \right\}$. Computing this upper bound on the Lipschitz functions tends to be hard nay impossible. We introduce $U\left(P,Q\right)$ the set of joint distributions whose marginals are $P$ and $Q$ and the type-$\ell$ Wasserstein distance: $$
W_\ell\left(P,Q\right) = \left\{\min_{\pi\in U\left(P,Q\right)}\int_{\Xi^2}\lVert \xi-\xi'\rVert^l \text{d}\pi\left(\xi,\xi'\right)\right\}^{1/l}.$$
Where $\lVert \cdot \rVert$ denotes the norm on $\Xi$ used to define $\mathcal{F}_1\left(\Xi\right)$. The next idea is to bound $\zeta_1$ by any $W_\ell$ which would be easier to compute.
\begin{align*}
    \zeta_1\left(P,Q\right)&= \sup_{F\in\mathcal{F}_1\left(\Xi\right)}\lvert \int_\Xi F\left(\xi\right)\text{d}P\left(\xi\right)-\int_\Xi F\left(\xi\right)\text{d}Q\left(\xi\right)\rvert, \\
    &=\mathcal{W}_1\left(P,Q\right), \quad \mathcal{W} \text{ being the dual of } W \text{, see \cite{peyre_computational_2019}[Chapter 6],}\\
    &= W_1\left(P,Q\right), \quad \text{due to the duality theorem of Kantorovich-Rubinstein.}
\end{align*}
In order to choose a $\ell$ that will simplify the work, $\ell\in\left\{1,2\right\}$ actually, we will prove that $W_1\left(P,Q\right)\leq W_\ell\left(P,Q\right)$ for any $\ell \geq 1$.  Let $\ell >1$, $q>1$ such that $\frac{1}{\ell}+\frac{1}{q}=1$ and $\pi\in U\left(P,Q\right)$:
\begin{align*}
W_1\left(P,Q\right)&\leq \int_{\Xi^2}\lVert \xi-\xi'\rVert \text{d}\pi\left(\xi\times\xi'\right),\: \pi \text{ may not be optimal,} \\ &\leq \left(\int_{\Xi^2} \lvert 1\rvert^{q}\text{d}\pi\left(\xi\times\xi'\right)\right)^{1/q} \left(\int_{\Xi^2}\lVert \xi-\xi'\rVert^\ell\text{d}\pi\left(\xi\times\xi'\right)\right)^{1/\ell}, \: \text{due to Hölder's inequality} \\
&=\left(\int_{\Xi^2}\lVert \xi-\xi'\rVert^\ell\text{d}\pi\left(\xi\times\xi'\right)\right)^{1/\ell}.
\end{align*}
We can now take the infimum over $\pi$ what gives:
$$
W_1\left(P,Q\right)\leq W_\ell\left(P,Q\right).
$$
For a two-stage linear model, we have finally proven that for $\ell\geq1$, there exists a constant $L>0$ such that:
$$
\lvert v\left(P\right)-v\left(Q\right)\rvert \leq LW_\ell\left(P,Q\right).
$$
Thereafter we will consider a finite distribution $P$ and attempt to find another distribution $Q$ with fewer atoms with the idea of minimizing the type-$\ell$ Wasserstein distance between them. This would control the term: $\lvert v\left(P\right)-v\left(Q\right)\rvert$. We will work with $\ell\in\left\{1,2\right\}$ for the theoretical part and $\ell=2$ for every simulation.

\section{Wasserstein distance with finite distribution}

In this part, we present tools and concepts useful for scenario reduction aswell as their guarantees and limits. Results derive from \cite{rujeerapaiboon_scenario_2022}. 

\subsection{Introduction to Wasserstein distance}
In the case of discrete distributions $P=\sum_{i\in I}p_i\delta_{x_i}$ and $Q=\sum_{j\in J}q_j\delta_{y_j}$. The type-$l$ Wasserstein distance between $P$ and $Q$ is defined as :  
\begin{definition}{Type-$\ell$ Wasserstein distance}
$$
d_\ell(P,Q)=\left(\min_{\pi\in\mathbb{R}_+^{\lvert I\rvert\times\lvert J\rvert}}\left\{ 
\sum_{i\in I}\sum_{j\in J}\pi_{ij}\lVert x_i-y_j\rVert^l \: \text{ : } \:  \begin{aligned}
& \sum_{j\in J}\pi_{ij}=p_i \: \forall i\in I \\
& \sum_{i\in I}\pi_{ij}=q_j \: \forall j\in J
\end{aligned}\right\}\right)^{1/\ell}.
$$
\end{definition}

\begin{remark}
    A more general definition can be written by replacing the norm with a specific distance between objects $x_i$ and $y_j$. 
\end{remark}
For $l\geq1$, this mathematical expression meets the requirements of a distance. Values are positive, symmetry is direct, definiteness requires more work but the idea is to start by proving that $d_\ell\left(P,Q\right)=0$ implies that $\text{supp}\left(P\right)=\text{supp}\left(Q\right)$, concluding will be easy after that. Triangle inequality derives from the gluing lemma, see e.g. \cite[Chapter 1]{peyre_computational_2019}. For $l\in\mathopen{]}0,1\mathclose{]}, d_\ell^\ell$ is a distance. A pro of this particular distance is to rely on the underlying structure meaning the norm. The philosophy of this distance is to measure the minimum cost of moving weights from $x_i$ to $y_j$ where costs are defined by the norm.
\newline

Let $\mathcal{P}_U(X,n)$ denote the set of all \emph{uniform} discrete distribution on $X\subset R^d$ with $n$ distinct scenarios (do not mind if I use scenarios instead of scenari) and $\mathcal{P}(X,m)$ denote the set of discrete distributions on $X\subset R^d$ with \emph{at most} m scenarios. 
\newline

As presented in the conclusion of \ref{stability}, we want to find a distribution that minimizes the Wasserstein distance over a particular set of distribution. Two types of reduction are typically presented. The continous scenario reduction problem :
$$
C_\ell(P,m)=\min_Q\left\{d_\ell(P,Q),\: Q\in\mathcal{P}(\mathbb{R}^d,m)\right\}, 
$$
and the discrete scenario reduction problem :
$$
D_\ell(P,m)=\min_Q\left\{d_\ell(P,Q),\: Q\in\mathcal{P}(\text{supp}(P),m)\right\}.
$$

Let $P$ be a finite distribution on $\RR^d$. Obviously $\text{supp}(P)\subset \mathbb{R}^d$ what leads to $C_\ell(P,m)\leq D_\ell(P,m)$. Discrete scenario reduction may be sometimes more suitable as continous scenario reduction can generate atoms that may be nonsensical in a particular context.

\subsection{Theoretical fundations of scenario reduction}
In this part, we provide bounds on the Wasserstein distance between an original distribution of $n$ atoms and its reduced one with $m<n$ atoms. Let $X\subset \RR^d$ and $\lVert \cdot \rVert$ be a norm on $X$. We restrict ourselves to distribution whose atoms lie within the unit ball and $\frac{1}{n}\sum_ix_i=0$. This is a legal assumption because of first repositionning a distribution does not affect $C_\ell$, second the positive homogeneity of $C_\ell$. Let $P=\sum_{i\in I}p_i\delta_{x_i}$, $\text{card}\left(I\right)=n<+\infty$, $1<m<n$, $\lambda\in\RR_+^*$ and $a\in\RR^d$:
$$
C_\ell\left(P,m\right)=\frac{1}{\lambda} C_\ell\left(\sum_{i\in I}p_i\delta_{\lambda x_i+a},m\right).
$$
We will work with uniform distribution $P\in\mathcal{P}_U\left(\RR^d,n\right)$ with the idea that the distributions derive from sampling and if not you can multiply atoms around a point of high probability. A goal of this part is to quantify and study the worst case of scenario reduction defined by:
$$    \bar{C}_\ell\left(n,m \right)=\left\{\max_{P\in\mathcal{P}_U(\mathbb{R}^d,n)}C_\ell(P,m)\: :\: \text{supp}(P)\in B(0,1)\right\}.
$$
Let us write a first bound that we will tighten thereafter, let $P\in\mathcal{P}_U(\mathbb{R}^d,n)$:
$$C_\ell\left(P,m\right)\leq C_\ell\left(P,1\right)\leq d_\ell\left(P,\delta_{0_{\RR^d}}\right)\leq 1.$$
As the right-hand side of the equation does not depend on $P$, one can take the upper bound over $P\in\mathcal{P}_U\left(\RR^d,n\right)$ and write that:
$$
\bar{C}_\ell\left(n,m \right)\leq1.
$$
Let $\mathfrak{P}(I,m)$ be the family of m-set partitions of $I$. Following \cite{rujeerapaiboon_scenario_2022}, we write $\{I_j\}$ an element of $\mathfrak{P}(I,m)$ and $I_j$ for $j\in\{1,..,m\}$ a set of the partition $\{I_j\}$.
\begin{theorem}{Reformulation}\label{theorem1}
$$C_\ell(P,m)=\min_{\{I_j\}\in \mathfrak{P}(I,m)}\left\{ \frac{1}{n}\sum_{j\in J}\min_{y_j\in\mathbb{R}^d}\sum_{i\in I_j}\lVert x_i-y_j\rVert^\ell \right\}^{1/\ell}.$$
\end{theorem}
\begin{remark}
    \rp{faire le lien avec la démo de coût optimaux}
\end{remark}
Assume that we use the euclidean norm. In the case $\ell=1$, $y_j^*$ is attained by any geometric median. Now if $\ell=2$, the inner minimum can be found easily, gradient of the inner function is well-defined and minimizing without constraints gives: $$y_j^*=\text{mean}\left(I_j\right)=\frac{1}{\lvert I_j\rvert}{\sum_{i\in I_j}x_i}.$$ 
\bibliography{biblio}
\bibliographystyle{alpha}
\end{document}
