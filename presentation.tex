\documentclass{beamer}
\usetheme[progressbar=frametitle, numbering=fraction, block=fill]{metropolis}           % Use metropolis theme
\usepackage{amsfonts}
\usepackage{amssymb}
\usepackage{amsthm}
\usepackage{color}
\usepackage{tikz}
\usepackage{algorithm, algorithmic}
\usepackage[colorinlistoftodos,bordercolor=orange,backgroundcolor=orange!20,linecolor=orange,textsize=scriptsize]{todonotes}
\usepackage{xcolor}
\renewcommand{\algorithmicrequire}{\textbf{Input:}}
\renewcommand{\algorithmicensure}{\textbf{Output:}}
\usepackage{graphicx} % for pdf, bitmapped graphics files+9-*9*-------------------------
\usepackage{epsfig} % for postscript graphics files
\usepackage{subcaption}
\usepackage{algorithm, algorithmic}

%% For causal graphs
\usetikzlibrary{shapes,decorations,arrows,calc,arrows.meta,fit,positioning}
\tikzset{
    -Latex,auto,node distance =0.5 cm and 1 cm,semithick,
    state/.style ={ellipse, draw, minimum width = 0.7 cm},
    point/.style = {circle, draw, inner sep=0.04cm,fill,node contents={}},
    bidirected/.style={Latex-Latex,dashed},
    el/.style = {inner sep=2pt, align=left, sloped},
}

%% Bigger progress bars
\makeatletter
\setlength{\metropolis@titleseparator@linewidth}{1pt}
\setlength{\metropolis@progressonsectionpage@linewidth}{2pt}
\setlength{\metropolis@progressinheadfoot@linewidth}{2pt}
\makeatother

% Macros
\newcommand{\np}[1]{\left(#1\right)}
\newcommand{\bp}[1]{\big(#1\big)}                           % Parenth\`{e}se big
\newcommand{\Bp}[1]{\Big(#1\Big)}                           % Parenth\`{e}se Big
\newcommand{\bgp}[1]{\bigg(#1\bigg)}                        % Parenth\`{e}se bigg
\newcommand{\Bgp}[1]{\Bigg(#1\Bigg)}                        % Parenth\`{e}se Bigg

\newcommand{\nc}[1]{\left[#1\right]}
\newcommand{\norm}[1]{\left\lVert#1\right\rVert}
\newcommand{\R}{\mathbb{R}}
\newcommand{\Q}{\mathbb{Q}}
\newcommand{\K}{\mathbb{K}}
\renewcommand{\C}{\mathbb{C}}
\newcommand{\Z}{\mathbb{Z}}
\newcommand{\N}{\mathbb{N}}
\newcommand{\E}{\mathbb{E}}
\newcommand{\X}{\mathbb{X}}
\renewcommand{\U}{\mathbb{U}}
\newcommand{\Sphere}{\mathbf{S}}
\newcommand{\Pro}{\mathbb{P}}
\DeclareMathOperator*{\argmin}{arg\,min}
\DeclareMathOperator*{\argmax}{arg\,max}
\newcommand*\diff{\mathop{}\!\mathrm{d}}
\DeclareMathOperator*{\opt}{opt}
\newcommand{\func}{\phi} % Fonctionnelle
\newcommand{\ufunc}{\overline{\phi}} %u(pper) function
\newcommand{\lfunc}{\underline{\phi}} %l(ower) function
\newcommand{\Func}{F}
\newcommand{\lFunc}{\underline{F}}
\newcommand{\uFunc}{\overline{F}}
\newcommand{\Funcb}{\mathbf{F}} % Ensemble specifiques de fonctionnelles
\newcommand{\lFuncb}{\underline{\mathbf{F}}}
\newcommand{\Funcbb}{\mathbb{F}} % Ensemble de fonctionnelles
\newcommand{\relint}{\mathop{\mathrm{ri}}}
\newcommand{\ce}[1]{[\![#1]\!]}
\newcommand{\Rb}{\overline{\R}}
\newcommand{\RbX}{\Rb^\X}
\newcommand{\Selection}[2][]{ % Selection function)
  \def\tst{#1}
  \def\testt{#2}
  \ifx\tst\empty
    \ifx\testt\empty
      S_t
    \else
      S_t\left( #2 \right)
    \fi
  \else
    S_t^{#1}\left( #2 \right)
  \fi
}
\newcommand{\lSelection}[2][w]{ % l(ower)Selection function)
  \def\tst{#1}
  \def\testt{#2}
  \ifx\tst\empty
    \ifx\testt\empty
      \underline{S}_t
    \else
      \underline{S}_t\left( #2 \right)
    \fi
  \else
    \underline{S}_t^{#1}\left( #2 \right)
  \fi
}
\newcommand{\uSelection}[2][w]{ % u(pper)Selection function)
  \def\tst{#1}
  \def\testt{#2}
  \ifx\tst\empty
    \ifx\testt\empty
      \overline{S}_t
    \else
      \overline{S}_t\left( #2 \right)
    \fi
  \else
    \overline{S}_t^{ #1 }\left( #2 \right)
  \fi
}
\newcommand{\lV}{\underline{V}} % l(ower)V
\newcommand{\uV}{\overline{V}} % u(pper)V
\newcommand{\supp}[1]{\mathrm{supp}\left(#1\right)} % support of a random variable
\newcommand{\cU}{\mathcal{U}_t^w} % constraint set-valued
\newcommand{\dom}{\mathrm{dom}\;}
\newcommand{\proj}{\pi} % euclidean projector

% In the following macros one optional parameter can be put between [].
\newcommand{\Qf}[3][w]{ % Q function
  \def\tst{#2}
  \ifx\tst\empty
    Q_{\func}^{#1}
  \else
    Q_{#2}^{#1}\left( #3 \right)
  \fi
}
\newcommand{\dyn}[2][w]{ % dynamic
  \def\tst{#2}
  \ifx\tst\empty
    f_t^{#1}
  \else
    f_t^{#1}\left(#2\right)
  \fi
}
\newcommand{\cost}[2][w]{ % cost function
  \def\tst{#2}
  \ifx\tst\empty
    c_t^{#1}
  \else
    c_t^{#1}\left(#2\right)
  \fi
}
\newcommand{\pB}[3][w]{ % p(ointwise)B(ellman operator)
  \def\tst{#2}
  \ifx\tst\empty
    \mathcal{B}_t^{#1}
  \else
    \mathcal{B}_t^{#1}\left( #2 \right) \left( #3 \right)
  \fi
}
\newcommand{\aB}[2]{ % a(verage)B(ellman operator)
  \def\tst{#1}
  \def\testt{#2}
  \ifx\tst\empty
    \mathfrak{B}_t
  \else
    \ifx\testt\empty
      \mathfrak{B}_t\left( #1 \right)
    \else
      \mathfrak{B}_t\left( #1 \right) \left(#2 \right)
    \fi
  \fi
}
\theoremstyle{plain}
\newtheorem{proposition}[theorem]{Proposition}
\newtheorem{assumption}{Assumption}

%%A Min-plus-SDDP Algorithm for Deterministic Multistage Convex Programming

\title{Scenario reduction : theory and algorithms}
\date{May 2024}
%\author{}
%\institute{}

\usepackage{tcolorbox}

\begin{document}

%%%% Beamer metropolis tweeks
\setbeamertemplate{section in toc}[sections numbered]
\setbeamertemplate{subsection in toc}[subsections numbered]
\setbeamerfont{section in toc}{series=\bfseries}
\setbeamerfont{footnote}{size=\scriptsize}

% \AtBeginSection[]{
%  \begin{frame}[noframenumbering]
%   \frametitle{Section content}
%   \tableofcontents[currentsection, sectionstyle=show/hide, currentsubsection, subsectionstyle=show/show/hide]
%  \end{frame}}
%%%%

\begin{frame}[plain]
 \maketitle
\end{frame}

\begin{frame}[noframenumbering]{Outline}
 \metroset{outer/numbering=none}
 \setbeamertemplate{section in toc}[sections numbered]
 \tableofcontents[hideallsubsections]
\end{frame}

\section{Theoretical fundations}

\begin{frame}{Motivation}
 Consider a stochastic program with a continuous distribution :
$$
\min_{x\in X}\int_\Xi f\left(x,\xi\right)P\left(d\xi\right).
$$
In order to solve this problem, we use numerical integral techniques which modify the objective function such as:
$$
\min_{x\in X}\sum_{i\in I}f\left(x,\xi_i\right)p_i
$$
with $p_i=P\left(\{\xi_i\}\right)$. Since $f$ can be expensive to compute, $\lvert I\rvert$ should be as small as we can, the role of scenario reduction is to provide a distribution that is close to $P=\sum_{i\in I}\delta_{\xi_i}p_i$ (in this case) but with a reduced number of atoms.
%https://www.mathematik.hu-berlin.de/~romisch/papers/Sapp09.pdf

 
\end{frame}


\begin{frame}{Wasserstein distance}
 Wasserstein distance defines closeness between probabilities using the structure $\lVert \cdot\rVert$ of the underlying space where atoms live. We write $\mathbb{P}=\sum_{i\in I}p_i\delta_{x_i}$, $\mathbb{Q}=\sum_{j\in J }q_j\delta_{y_j}$. The type-$l$ Wasserstein distance ($l\geq1$) between $\mathbb{P}$ and $\mathbb{Q}$ is : 
 \[
d_l(\mathbb{P},\mathbb{Q})^l=\min_{\pi\in\mathbb{R_+^{|I|\times|J|}}}\left\{ 
\sum_{i\in I}\sum_{j\in J}\pi_{ij}\lVert x_i-y_j\rVert^l \: \text{ : } \:  \begin{aligned}
& \sum_{j\in J}\pi_{ij}=p_i \: \forall i\in I \\
& \sum_{i\in I}\pi_{ij}=q_j \: \forall j\in J
\end{aligned}\right\}.
\]
\newline

Basically, the goal is to find the optimal way to move the masses from the atoms of $\mathbb{P}$ to the atoms of $\mathbb{Q}$ such that the total transportation cost is minimized.
\end{frame}

\begin{frame}{Notation}
\[
d_l(\mathbb{P},\mathbb{Q})^l=\min_{\pi\in\mathbb{R_+^{|I|\times|J|}}}\left\{ 
\sum_{i\in I}\sum_{j\in J}\pi_{ij}\lVert x_i-y_j\rVert^l \: \text{ : } \:  \begin{aligned}
& \sum_{j\in J}\pi_{ij}=p_i \: \forall i\in I \\
& \sum_{i\in I}\pi_{ij}=q_j \: \forall j\in J
\end{aligned}\right\}.
\]
    $\pi^*$ is the transportation plan and the constraints are named mass conservation.
    \newline
    
    $\mathcal{P}_E(X,n)$ denotes the set of all \textbf{uniform} discrete distribution on $X\subset R^d$ with \textbf{exactly} $n$ distinct scenarii and $\mathcal{P}(X,m)$ denotes the set of discrete distributions on $X\subset R^d$ with \textbf{at most} m scenarii. 
\end{frame}

\begin{frame}{What do we want to solve}
    The continuous scenario reduction problem : 
$$
C_l(\mathbb{P},m)=\min_\mathbb{Q}\left\{d_l(\mathbb{P},\mathbb{Q}),\: \mathbb{Q}\in\mathcal{P}(\mathbb{R}^d,m)\right\}. 
$$
The discrete scenario reduction problem :
$$
D_l(\mathbb{P},m)=\min_\mathbb{Q}\left\{d_l(\mathbb{P},\mathbb{Q}),\: \mathbb{Q}\in\mathcal{P}(\text{supp}(\mathbb{P}),m)\right\}.
$$
\end{frame}

\section{Algorithms}
\begin{frame}{K-means clustering}
Generalization of k-means clustering to arbitrary norm and power $l$. Gives an approximation of $C_l\left(\mathbb{P},k\right).$
\begin{algorithm}[H]
    \caption{k-means clustering for $C_l\left(\mathbb{P},m\right)$}
    1. Initialize the reduced set $R=\{y_1,...,y_m\} \subseteq \text{supp}\left(\mathbb{P}\right)$ arbitrarily. \\ 2. Let $\{I_j\}\in\mathfrak{P}\left(I,m\right)$ be any partition whose sets $I_j, j\in J$, contain all atoms of supp$\left(P\right)$ that are closest to $y_j$ (ties may be broken arbitrarily). \\ 3. For each $j\in J$, update $y_j$ as : $$y_j \gets \argmin_{y\in\RR^d} \left\{ \sum_{i\in I_j}\lVert x_i-y\rVert^l\right\}$$ \\ 4. Repeat Steps 2 and 3 until the reduced set $R$ no longer changes.
\end{algorithm}
\end{frame}

\begin{frame}{Dupačová et al.}
Dupačová et al. \cite{dupacova_scenario_2003} provides a greedy algorithm for $D_l\left(\mathbb{P},m\right)$. Generally, the distribution obtained isn't a solution of the discrete scenario reduction problem.
\begin{algorithm}[H]
  \caption{Dupačová et al.}\label{dupacova}
  1. Initialize the set of atoms in the reduced set as $R\gets \emptyset.$ \\ 2. Select the next atom to be added to the reduced set as $$ y\in\argmin_{y\in \text{supp}\left(\mathbb{P}\right)}D_l\left(\mathbb{P},R\cup\{y\}\right).
  $$
  Update $R\gets R\cup \{y\}$.\\ 3. Repeat Step 2 until $\lvert R\rvert=m$.
\end{algorithm}
\end{frame}

\begin{frame}{Step 2 ?}
    Step 2 of the above algorithm requires to compute :
    $$
    D_l\left(\mathbb{P},R\cup\{y\}\right).
    $$
    If we write $R\cup\{y\}=\{y_j\}_{j\in J}$, this problem can be written as : 
    $$
    \min_{q\geq 0}d_l\left(\mathbb{P}, \sum_{j\in J}q_j\delta_{y_j}\right).
    $$
    We'll talk about it at the end of the presentation.
\end{frame}

\begin{frame}{Local search algorithm (1/2)}
    \begin{algorithm}[H]
    \caption{Local search algorithm for $D_l\left(\mathbb{P},m\right)$}\label{Local search}
    1. Initialize the reduced set $R\subseteq \text{supp}\left(\mathbb{P}\right), \lvert R\rvert = m$, arbitrarily. \\ 2. Select the next change to be applied to the reduced set as 
    $$
    \left(y,y'\right)\in\argmin\left\{D_l\left(\mathbb{P},R\cup\{y\}\setminus \{y'\}\right) : \left(y,y'\right)\in\text{supp}\left(\mathbb{P}\setminus R\right)\times R\right\}.
    $$
    Update $R\gets R\cup \{y\}\setminus \{y'\}$ if $D_l\left(\mathbb{P}, R\cup\{y\}\setminus \{y'\} \right)<D_l\left(\mathbb{P},R\right).$ \\ 3. Repeat Step 2 until no further improvement is possible.
    \end{algorithm}
    This algorithm can make you think of the Lin-Kernighan heuristic for the TSP when you have to visit only $m$ towns out of the $n$.
\end{frame}

\begin{frame}{Local search algorithm (2/2)}
    Same issue for Step 2, we have to compute :
    $$
    \min_{q\geq 0}d_l\left(\mathbb{P}, \sum_{j\in J}q_j\delta_{y_j}\right).
    $$
    Unlike Dupačová et al., this algorithm has theoretical guarantees about the output, it'll always be close to the optimal output, the approximation ratio is bounded. 
\end{frame}

\begin{frame}{Explicit value for Step 2.}
    What maybe exists but we haven't found on article talking about these algorithms \cite{rujeerapaiboon_scenario_2022}, \cite{bertsimas_optimization-based_2023} is an explicit value for step 2, meaning : 
    $$
    \min_{q\geq 0}d_l\left(\mathbb{P}, \sum_{j\in J}q_j\delta_{y_j}\right)= \min_{q\geq 0}d_l\left(\sum_{i\in I}\delta_{x_i}p_i, \sum_{j\in J}q_j\delta_{y_j}\right).
    $$
    If my proof works indeed : 
$$
    \min_{q\geq 0}d_l^l\left(\sum_{i\in I}\delta_{x_i}p_i, \sum_{j\in J}q_j\delta_{y_j}\right)=\sum_{i\in I}p_i\min_{k\in J}\lVert x_i-y_k\rVert^l
$$
The proof also gives a transporation plan and a distribution $\mathbb{Q}$ reaching this value. Then, we wouldn't rely on solving optimal transport problems in order to compute for Dupačová et al' algorithm and local search algorithm.
\end{frame}

\begin{frame}{MILP reformulation}
    
\end{frame}
\bibliography{biblio}
\bibliographystyle{alpha}

\end{document}
