\documentclass{amsart}
\usepackage[T1]{fontenc}
\usepackage[utf8]{inputenc}
\usepackage{amsmath}
\usepackage{amsfonts}
\usepackage{amssymb}
\usepackage[left=2cm,right=2cm,top=3cm,bottom=3cm]{geometry}
\usepackage{graphicx}
\usepackage{hyperref}
\usepackage{cleveref}
\usepackage{bm}
\usepackage[colorinlistoftodos,bordercolor=orange,backgroundcolor=orange!20,linecolor=orange,textsize=scriptsize]{todonotes}
\definecolor{red}{cmyk}{0,1,.8,0}
\definecolor{blue}{rgb}{0,0,1}

\usepackage{xcolor}
\hypersetup{
    colorlinks,
    linkcolor={red!50!black},
    citecolor={blue!50!black},
    urlcolor={blue!80!black}
}

\usepackage{subcaption}
\ifdefined\theorem\else \newtheorem{theorem}{Theorem}\fi
\ifdefined\proposition\else \newtheorem{proposition}[theorem]{Proposition}\fi
\ifdefined\definition\else \newtheorem{definition}[theorem]{Definition}\fi
\ifdefined\lemma\else \newtheorem{lemma}[theorem]{Lemma}\fi
\ifdefined\corollary\else \newtheorem{corollary}[theorem]{Corollary}\fi
\ifdefined\remark\else \newtheorem{remark}[theorem]{Remark}\fi
\ifdefined\assumption\else \newtheorem{assumption}{Assumption}\fi
\ifdefined\example\else \newtheorem{example}{Example}\fi

\newcommand{\argmin}{\mathop{\arg\min}}

\newcommand{\nb}[3]{
		{\colorbox{#2}{\bfseries\sffamily\tiny\textcolor{white}{#1}}}
		{\textcolor{#2}{\text{$\blacktriangleright$}{\textcolor{#2}{#3}}\text{$\blacktriangleleft$}}}}
\newcommand{\rp}[1]{\nb{RP}{red}{#1}}
\newcommand{\af}[1]{\nb{AF}{blue}{#1}}
\newcommand{\bt}[1]{\nb{BT}{orange}{#1}}
\newcommand{\RR}{\mathbb{R}}

\title{Scenario reduction}

\begin{document}
\maketitle
In this draft, I aim to sum up \cite[2022]{rujeerapaiboon_scenario_2022}

\section{Introduction}
\subsection{Context}
The goal of scenario reduction is to compute a distribution that is "close" to an initial distribution but with a reduced number of atoms. In this part, we present tools and concepts useful for scenario reduction. We define closeness between two objects through distances. When objects are distributions, the Wasserstein distance is a good candidate. In the case of discrete distributions $\mathbb{P}=\sum_{i\in I}p_i\delta_{x_i}$ and $\mathbb{Q}=\sum_{j\in J}q_j\delta_{y_j}$, one can note that $x_i$ and $y_j$ must live in the same normed space ($\lVert.\rVert$ denotes the norm). The type-$l$ Wasserstein distance between $\mathbb{P}$ and $\mathbb{Q}$ is defined as :  
\[
d_l(\mathbb{P},\mathbb{Q})=\left(\min_{\pi\in\mathbb{R_+^{|I|\times|J|}}}\left\{ 
\sum_{i\in I}\sum_{j\in J}\pi_{ij}\lVert x_i-y_j\rVert^l \: \text{ : } \:  \begin{aligned}
& \sum_{j\in J}\pi_{ij}=p_i \: \forall i\in I \\
& \sum_{i\in I}\pi_{ij}=q_j \: \forall j\in J
\end{aligned}\right\}\right)^{1/l}.
\]
The parameter $l$ is said to be greater than or equal to 1. \rp{don't really know why it's not only greater than 0} \bt{To prove that $d^\ell$ is distance, the difficult part is the triangle inequality. It is a distance for $\ell \geq 1$ and the proof of the triangle inequality is here called the "gluing lemma" (Peyré-Cuturi chapter 1 for our "simple" case). This gluing lemma uses Minkowski's inequality in the case $\ell\geq1$ so that's the reason why. 
It is still an interesting question for $0 < \ell \leq 1$. We have $d_\ell^\ell$ (and not $d\ell$ itself) is a distance as $(x,y) \mapsto \lVert x - y \rVert^\ell$ is already a distance when $0 < \ell \leq 1$ thanks to Minkowski's inequality in the case $\ell \leq 1$.
}. See \ref{compute} in order to have an idea on how to compute the distance of Wasserstein between two distributions and more precision.
\newline

$\mathcal{P}_E(X,n)$ denotes the set of all \textbf{uniform} discrete distribution on $X\subset R^d$ with \textbf{exactly} $n$ distinct scenarii and $\mathcal{P}_(X,m)$ denotes the set of discrete distributions on $X\subset R^d$ with \textbf{at most} m scenarii. In the article, one assume that $\mathbf{P}\in\mathcal{P}_E(X,n)$. What is true when you obtain the distribution $\mathbb{P}$ via sampling, every issue has the probability of $\frac{1}{n}$ and if a same value occurs twice or more, they say we can decompose the distribution with two or more close atoms.
\newline

We want to find a distribution that minimizes the Wasserstein distance over a particular set of distribution just like we can do in $\RR^d$ when we want to project a point over a subspace. Two types of reduction are typically presented. The continous scenario reduction problem :
$$
C_l(\mathbb{P},m)=\min_\mathbb{Q}\left\{d_l(\mathbb{P},\mathbb{Q}),\: \mathbb{Q}\in\mathcal{P}(\mathbb{R}^d,m)\right\}. 
$$
The discrete scenario reduction problem :
$$
D_l(\mathbb{P},m)=\min_\mathbb{Q}\left\{d_l(\mathbb{P},\mathbb{Q}),\: \mathbb{Q}\in\mathcal{P}(\text{supp}(\mathbb{P}),m)\right\}.
$$
Let $\mathbb{P}$ be a finite distribution on $\RR^d$, obviously $\text{supp}(\mathbb{P})\subset \mathbb{R}^d$ leads to $C_l(\mathbb{P},m)\leq D_l(\mathbb{P},m)$. Discrete scenario reduction problem has been more studied as it uses atoms of $\mathbb{P}$ that have a physical reality (if $\mathbb{P}$ represents the distribution of a real event).

\section{Fundamental limits of scenario reduction}
\subsection{Bounds}
Here, we try to provide bounds on the value of the Wasserstein distance in order to guarantee closeness between the reduced distribution and the original one. We work with $\lVert.\rVert_2$ and atoms within the unit ball because of the positive homogeneity of the Wasserstein distance $C_l(\mathbb{P}^\lambda,m)=\lambda \cdot C_l(\mathbb{P},m)$ for $\lambda\in\mathbb{R}_+$ and $\mathbb{P}^\lambda=\sum_{i\in I}p_i\delta_{\lambda\cdot x_i}$. See \ref{positive h} for a proof. \rp{not obvious tbh} Generally, we have that $$\forall \lambda\in\mathbb{R}, C_l(\mathbb{P}^\lambda,m)=|\lambda|C_l(\mathbb{P},m).$$ 
With this homogeneity, working in the unit ball is a legit assumption as we can scale distributions w.l.o.g. Another goal of this part is to quantify and study the worst case of scenario reduction defined by
$$
\bar{C}_l\left(n,m \right)=\left\{\max_{\mathbb{P}\in\mathcal{P}_E(\mathbb{R}^d,n)}C_l(\mathbb{P},m)\: :\: \text{supp}(\mathbb{P})\in B(0,1)\right\}.
$$
The case that gives the highest value of optimal Wasserstein distance. The goal is to find an upper bound better than 1 for $\bar{C}_l\left(n,m \right)$. First, we prove that : $\bar{C}_l\left(n,m \right)\leq 1$. By definition, we have that  
$$d_l(\mathbb{P},\delta_0)=\left(\min_{\pi\in\mathbb{R_+^n}}\left\{ 
\sum_{i=1}^n\pi_i\lVert x_i\rVert^l \: \text{ : } \:  \begin{aligned}
& \pi_{i}=p_i, \: \forall i\in I \\
& \sum_{i= 1}^n\pi_{i}=1 
\end{aligned}\right\}\right)^{1/l} \leq 1.$$
Here we used that $\lVert x_i\rVert\leq 1$. Again by definition, we have that
$$
C_l(\mathbb{P},m)\leq C_l(\mathbb{P},1)\leq d_l(\mathbb{P},\delta_0)\leq 1.
$$
1 doesn't depend on $\mathbb{P}$, so we can go to the upper bound for $\mathbb{P}\in\mathcal{P}_E(\mathbb{R}^d,n), \text{supp}(\mathbb{P})\in B(0,1)$ and we proved that $\bar{C}_l\left(n,m \right)\leq1$ \rp{this bound even works for non uniform discrete distributions and doesn't matter on the norm we use as long as $\text{supp}(\mathbb{P})\in B(0,1)$ for the norm}
The aim of what's next is to tighten that bound for particular $l$ and with $\lVert.\rVert_2$
\newline

$\mathfrak{P}(I,m)$ is the family of m-set partitions of $I$.
\begin{theorem}{Reformulation}
$$C_l(\mathbb{P},m)=\min_{\{I_j\}\in \mathfrak{P}(I,m)}\left\{ \frac{1}{n}\sum_{j\in J}\min_{\zeta_j\in\mathbb{R}^d}\sum_{i\in I_j}\lVert x_i-\zeta_j\rVert^l \right\}^{1/l}.$$
\end{theorem}

For $l=2$, the inner minimum, $\zeta_j^*=\frac{1}{|I_j|}\sum_{i\in I_j}x_i$. For $l=1$, $\zeta_j^*$ is attained by any geometric median \rp{I can't find anything for that notion}

\begin{theorem} Type-2 Wasserstein distance upper bound
    $$\bar{C}_2\left(n,m \right)\leq \sqrt{\frac{n-m}{n-1}}.$$
\end{theorem}
\rp{The use of $\tau$ in the proof is legit as $\tau \leq \sum_{j\in J}\sum_{i\in I_j}\lVert x_i-\text{mean}(I_j)\rVert_2^2,\:\forall \{I_j\}$ means that $\tau\leq\min_{\{I_j\}}\sum_{j\in J}\sum_{i\in I_j}\lVert x_i-mean(I_j)\rVert_2^2$ but as we want to maximize the function $\tau$ will be the highest he can, so it'll be equal to the minimum, thereafter it gets technical but understandable.}

End of Lemma 1 is also technical as the set of solutions of a convex problem is convex, which is necessary to say that : $S=\frac{1}{n!}\sum_{\sigma\in\mathfrak{S}}S^\sigma$ is also a solution. This matrix S is indeed invariant under permutations. Lemma 2 is easy but an easier proof that doesn't use circulant matrix is : 
$$\chi_A(X)=\text{det}(\begin{bmatrix}
    X- \alpha - \beta & -\beta&\hdots&-\beta \\
    -\beta & \ddots &\ddots&\vdots \\
    \vdots &\ddots &\ddots&   \\
    &&&-\beta\\
    -\beta&\hdots&-\beta&X-\alpha-\beta
\end{bmatrix})=(X-\alpha-n\beta)\text{det}(A_1).$$
where $A_1$ is the matrix A but with the first column replaced by 1. The first equality rises from adding every column to the first one, so the first column is a column of $X-\alpha-n\beta$. What we have to do next is do add $\beta\cdot C_1$ to every other column, what gives us a lower triangular matrix $diag(1,X-\alpha,...,X-\alpha)$, and we have $$
\chi_A(X)=(X-\alpha-n\beta)(X-\alpha)^{n-1}$$

Then they show a distribution \textbf{that equals the upper bound for \boldmath{$d\geq n-1$}(meaning that the upper bound is optimal)} under these assumptions.

Same thing can be proven for $\lVert.\rVert_1$
$$
\bar{C}_1\left(n,m \right)\leq\bar{C}_2\left(n,m \right) \leq\sqrt{\frac{n-m}{n-1}}.
$$
\begin{proof}
    Geometric as $$\sum_{i\in I_j}\lVert x_i-\text{gmed}(I_j)\rVert_2\leq \sum_{i\in I_j}\lVert x_i-\text{mean}(I_j)\rVert_2, \: \forall j\in J.$$ 
\end{proof}
Maybe the bound can be tightened because the proof only proves : $$\bar{C}_1\left(n,m \right)\leq\bar{C}_2\left(n,m \right).$$
Unlike $\bar{C}_2\left(n,m \right)$, we also get a lower bound on $\bar{C}_1\left(n,m \right)$: $$\bar{C}_1\left(n,m \right)\geq\sqrt{\frac{(n-m)(n-m+1)}{n(n-1)}}\geq \frac{n-m}{n-1}.$$
Now the right hand side equals $\bar{C}_2^2(n,m)$ whenever $d\geq n-1$. Then we can write the following proposition :
\begin{proposition}Whenever $d\geq n-1$,  
$$\bar{C}_2^2\left(n,m \right)\leq \bar{C}_1\left(n,m \right)\leq\bar{C}_2\left(n,m \right).$$
\end{proposition}
\subsection{Use of bounds}
One can use the bounds we found above in order to guarantee that the reduced distribution is close enough (to be defined...) to the original one. For $l\in\{1,2\}$ and a large $n$ : $$\bar{C}_l\left(n,m \right)\leq \sqrt{\frac{n-m}{n-1}}=\sqrt{\frac{1-\frac{m}{n}}{1-\frac{1}{n}}}\underset{n \to +\infty}{=}\sqrt{1-\frac{m}{n}}(1+\frac{1}{2n}+O(\frac{1}{n^2}))\approx \sqrt{1-\frac{m}{n}}.$$
This latest result and the positive homogeneity of $C_l$ provides an upper bound that depends only on the ratio $p=\frac{m}{n}$ for all kind of discrete distribution $\mathbb{P}=\frac{1}{n}\sum_{i\in I}\delta_{x_i}$. Denote by $r\geq 0$ and $\mu\in\mathbb{R}^d$ the radius and the center of any  (ideally the smallest) ball enclosing $\text{supp}(\mathbb{P})$, we have that :
$$
C_l(\mathbb{P},m)=r\cdot C_l(\frac{1}{n}\sum_{i\in I}\delta_{\frac{x_i-\mu}{r}},m)\leq r\cdot \bar{C_l\left(n,m \right)}\leq r\cdot \sqrt{1-p}.
$$
One can think that $C_l(\mathbb{P},m)$ is much smaller than $r\cdot \sqrt{1-p}$ but in fact it often happens even when $\left\{x_i\right\}$ are sampled from a normal distribution for example, see \textbf{Proposition 3} of \cite{rujeerapaiboon_scenario_2022}.
As an example, we can choose : $\mu=\frac{1}{n}\sum_{i\in I}x_i$ and $r=\max_i\left\{{\lVert x_i-\mu\rVert}\right\}$. One can easily verify that this couple defines a ball enclosing $\text{supp}\left(\mathbb{P}\right)$. Then it follows that : 
$$
C_l(\mathbb{P},m)\leq \max_i\left\{{\lVert x_i-\mu\rVert}\right\}\sqrt{1-p}.
$$
what depends only on $\mathbb{P}$. \rp{maybe that helps a very little the article}. A good idea \rp{imo} is to find the smallest bound for a given distribution $\mathbb{P}$ and a given norm $\lVert.\rVert$, so we can have a better upper bound. One must find the Chebychev center : 
\begin{align*}
    &\min_{r\in\RR^+,\mu\in\RR^d} \quad r\\
    &\text{s.t.}\quad \lVert\frac{x_i-\mu}{r}\rVert \leq 1, \;\forall i\in I.
\end{align*}
See \ref{chebyshev} but geometrically what I wrote seemed to be a good candidate.
\newline

\section{What about discrete scenario reduction}
Naturally, the continuous scenario reduction gives a smaller distance than the one computed by the discrete scenario reduction. We'll try to quantify and compare the gap between the solution of both problems.


\newpage

\appendix
\section{Computing Wasserstein distance}
\label{compute}
Let $\mathbb{P}=\sum_{i\in I}\delta_{x_i}p_i, \mathbb{Q}=\sum_{j\in J}\delta_{y_j}q_j$ two finite distributions. Computing the distance is equivalent to solving a linear program. Thus it is "easily" computable (with the simplex or with better-fitting techniques as it is a min-cost transportation problem).
If we write $n=|I|, m=|J|$ and :
$$
\pi=\begin{bmatrix}
    \pi_{11}& \hdots&\pi_{1m}&\pi_{21}&\hdots&\pi_{2m}&\hdots\pi_{nm}
\end{bmatrix}^T.
$$
Than the constraints can be visualized with the matrix $A=\begin{pmatrix}
    A_I \\
    A_J
\end{pmatrix}$ with $A_I$ $k^{th}$ row (out of the $n$ rows) being : 
$$\begin{bmatrix}
    0&\hdots&0&1&\hdots&1&0&\hdots&0
\end{bmatrix}.$$
with $m\times(k-1)$ zeros at the beginning, then $m$ ones and then zeros. 
\newline
Similarly $A_J$ $k^{th}$ row (out of the $m$ rows) is full of zeros except at position $k+(j-1)\times m, \forall j\in J$ where you'll find ones. This way, it's easy to find a distance between two finite distributions.
\newline

One more thing we can say is that $rk(A)=n+m-1$. The rows of A are linearly dependant : $L_1+...+L_n-L_{n+1}-...-L_{n+m}=0$. Then if we take a linear combination of rows $A_1,...,A_{n+m-1}$ that is equal to zero, we have that $$
\sum_{i=1}^{n+m-1}\lambda_iL_i^T=\begin{bmatrix}
    \lambda_1+\lambda_{n+1}\\ \vdots\\ \lambda_1+\lambda_{n+m-1}\\ \lambda_{1} \\ \lambda_2+\lambda_{n+1} \\ \vdots \\ \lambda_2+\lambda_{n+m-1}\\ \lambda_2  \\ \vdots
\end{bmatrix}=0_{\RR^{nm}}.
$$
Directly, we see that $\forall i \in [\![1;n]\!], \lambda_i=0$ when we look at the $(m\times i)^{th}$ coefficient and then $\forall k\in[\![1;n+m-1]\!], \lambda_k=0$ is easy to see. Then $rk(A)=n+m-1$. \rp{kind of useless as we know that the number of variables in basis is $n+m$ (number of constraints) but can be the starting point of sparsity explanation if I study regularization as Benoît mentionned}
\section{Positive homogeneity of Wasserstein distance}
\label{positive h}
 Let $\mathbb{Q}$ a distribution with at most $m$ atoms. We have that
\begin{align*}
    d_l(\mathbb{P}^\lambda,\mathbb{Q})^l&=\min_{\pi\in\mathbb{R_+^{|I|\times|J|}}}\left\{ 
\sum_{i\in I}\sum_{j\in J}\pi_{ij}\lVert\lambda x_i-y_j\rVert^l \: \text{ : } \:  \begin{aligned}
& \sum_{j\in J}\pi_{ij}=p_i \: \forall i\in I \\
& \sum_{i\in I}\pi_{ij}=q_j \: \forall j\in J
\end{aligned}\right\} \\&=\lambda^l\cdot\min_{\pi\in\mathbb{R_+^{|I|\times|J|}}}\left\{ 
\sum_{i\in I}\sum_{j\in J}\pi_{ij}\lVert x_i-\frac{1}{\lambda}y_j\rVert^l \: \text{ : } \:  \begin{aligned}
& \sum_{j\in J}\pi_{ij}=p_i \: \forall i\in I \\
& \sum_{i\in I}\pi_{ij}=q_j \: \forall j\in J
\end{aligned}\right\} \\
&=\lambda^l \cdot d_l(\mathbb{P},\mathbb{Q}^{\frac{1}{\lambda}})^l.
\end{align*}
We've proved that $d_l(\mathbb{P}^\lambda,\mathbb{Q})=\lambda\cdot d_l(\mathbb{P},\mathbb{Q}^{\frac{1}{\lambda}})$. Now, we have $$C_l(\mathbb{P}^\lambda,m)\leq d_l(\mathbb{P}^\lambda,\mathbb{Q})=\lambda\cdot d_l(\mathbb{P},\mathbb{Q}^{\frac{1}{\lambda}})$$
But the left hand side doesn't depend on $\mathbb{Q}$ any longer so we can minimize the right hand side :
$$
C_l(\mathbb{P}^\lambda,m)\leq \lambda\cdot C_l(\mathbb{P},m).
$$
Similarly we can prove : $C_l(\mathbb{P}^\lambda,m)\geq \lambda\cdot C_l(\mathbb{P},m)$, both results prove the positive homogeneity of the Wasserstein distance.

\section{Chebyshev center problem}
\label{chebyshev}
The Chebyshev problem is :
\begin{align*}
    &\min_{r,\mu} \quad r\\
    &\text{s.t.}\quad \lVert x_i-\mu\rVert^2-r^2 \leq 0, \;\forall i\in I.\\
    &-r\leq0
\end{align*}
We can write the Lagrangian of this problem : 
$$
\mathcal{L}(r,\mu,\lambda)=r+\lambda_0r+\sum_{i\in I}\lambda_i(\lVert x_i-\mu\rVert^2-r^2)
$$
KKT conditions provide at optimality : \begin{align*}
    \lambda\geq0\\
    \nabla_r\mathcal{L}(r,\mu,\lambda)=0\\
    \nabla_\mu\mathcal{L}(r,\mu,\lambda)=0\\
    \forall i\in I,\;\lambda_i(\lVert x_i-\mu\rVert^2-r^2)=0 \\
    \lambda_0r=0
\end{align*}
The gradient isn't always defined according to the norm but when it's $\lVert.\rVert_2$ : 
\begin{align*}
    \lambda\geq0\\
    1+\lambda_0-2r\sum_{i\in I}\lambda_i=0\\
    \sum_{i\in I}\lambda_i(x_i-\mu)=0\\
    \forall i\in I,\;\lambda_i(\lVert x_i-\mu\rVert^2-r^2)=0 \\
    \lambda_0r=0
\end{align*}
\rp{Can't conclude right now with these conditions, need to think more I guess, maybe I can start by proving that $0_{\RR^d},max\lVert x_i\rVert$ is the optimal couple when the family is centered and then extending it to every family by centering the family with $x_i-\mu$ with $\mu$ being the mean of the family}


\bibliography{biblio}
\bibliographystyle{alpha}
\end{document}
